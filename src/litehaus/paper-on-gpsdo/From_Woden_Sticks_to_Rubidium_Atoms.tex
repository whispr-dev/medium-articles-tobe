\documentclass[11pt]{article}

\usepackage[T1]{fontenc}
\usepackage[utf8]{inputenc}
\usepackage[a4paper,margin=1in]{geometry}

\usepackage{microtype}
\usepackage{hyperref}
\usepackage{enumitem}
\usepackage{xurl}

\setlist{nosep}

\begin{document}

\begin{center}
{\LARGE \textbf{FROM WOOODEN STICKS TO RUBIDIUM ATOMS:}}\\
\vspace{0.5em}
{\large \textbf{An even briefer and weirder history of time than Steven Hawking's}}
\end{center}

\bigskip
\hrule
\bigskip

\section*{Prertaining to Time.}
Terrestrial Dynamical Time / time-scale overview (NASA) \cite{ref1}

\section*{From Shadow-Sticks to ``1900 January 0'': A Cozy (and Slightly Unhinged) History of Measuring Time}
\textbf{L}et’s start with a confession: timekeeping is one of humanity’s longest-running DIY projects, and we’ve been shipping ``v1.0'' patches for at least ten thousand
years. Sometimes it’s awe-inspiring. Sometimes it’s basically ``I poked a stick in the ground and the shadow moved, therefore… lunch?'' And somehow, that messy
lineage eventually leads to a definition of the second that references a specific year (1900) and a very particular kind of year (the tropical year) sliced
into a frighteningly precise fraction.
So here’s the story—warm, human, slightly storm-battered—of how we got from ancient stones and sky-gazing to the moment scientists effectively said:
``Right, enough with the wobbly Earth, let’s define time using the orbit instead.'' \cite{ref1}

\subsection*{The First Clocks Were Just… Being Alive}
Before clocks, humans measured time with bodies and seasons: hunger, fatigue, migration, the return of animals, the ripening of plants. No one needed ``3:17 PM''
in the Paleolithic; you needed ``the light is dying—get back before something with teeth notices.''
But once you start farming—once you need to anticipate weather and coordinate labor—time becomes less vibes and more infrastructure. That’s when the sky stops
being just pretty and starts being a schedule.

\subsection*{Göbekli Tepe: The Stone Age Looks Up (and Maybe Counts Days?)}
\textbf{G}öbekli Tepe—this astonishing, ancient site in southern Turkey—has been stirring the timekeeping pot lately because researchers argue some carvings may
encode calendrical knowledge.
In 2024, University of Edinburgh reporting on work published in Time and Mind described an interpretation where V-shaped symbols on pillars may represent days,
adding up to a 365-day solar calendar composed of 12 lunar months plus 11 extra days.
If that interpretation holds (and it is very much a debated, interpretive space rather than ``case closed''), it suggests people were not only tracking seasons
but were symbolically encoding cycles of the Sun and Moon with real numerical discipline.

\textbf{E}ven the idea of a ``special day'' aligned with the summer solstice appears in this narrative—suggesting a kind of seasonal anchor point, not unlike the way later
civilizations built ritual and agriculture around solstices.
Now, the skeptical-but-hopeful takeaway here isn’t ``Stone Age people had iPhones.'' It’s that the impulse behind timekeeping—turning the sky into a stable
reference—might be older than our neat textbook timelines.

\subsection*{Shadow-Sticks, Obelisks, and the Birth of ``Daytime Math''}
\textbf{O}nce civilizations got going in earnest, timekeeping went from symbolic to practical. Egypt is a big chapter here.
One widely cited tradition places early sundial-like instruments—simple gnomons and shadow-measuring devices—thousands of years BCE, with Egyptian developments
often cited around the 4th millennium BCE.
By around 1500 BCE, we even have surviving physical evidence of more formal sundials, with the day subdivided into parts—timekeeping not just for ritual, but
for organizing work.
What matters isn’t the exact brand of ``shadow clock.'' It’s the conceptual leap: daylight can be cut into regular-ish segments. That’s the beginning of time as
a countable resource.

\section*{When the Sun Won’t Cooperate: Water Clocks and Night-Time Hours}
\textbf{T}he Sun is great until it’s cloudy or—small inconvenience—it’s night. So people got clever.
Water clocks (clepsydrae) emerged as a way to track hours in darkness, often using a vessel with a small hole so water could drain at a controlled rate, with
markings indicating the passing of hours.
This was timekeeping’s first big ``engineering reality check'': if your method depends on perfect weather and daylight, it’s not a system—it’s a suggestion.

\subsection*{The Sexagesimal Plot Twist: Why We Still Live Inside Base-60}
\textbf{N}ow we need to talk about the weird, beautiful number 60—the reason minutes have 60 seconds and hours have 60 minutes.
The sexagesimal (base-60) tradition is associated with ancient Mesopotamian mathematics and astronomy, and it’s one of the most persistent design decisions
in human history.
Sixty is absurdly practical because it has lots of divisors (2, 3, 4, 5, 6… and more), so it plays well with fractions long before everyone’s carrying a
calculator in their pocket.
This is how timekeeping starts to feel like engineering: choices made for computational convenience become cultural reality for millennia.

\subsection*{From ``Temporal Hours'' to Equal Hours: The Mechanical Clock Revolution}
\textbf{E}arly timekeeping often used temporal hours—hours that changed length depending on season (longer daylight hours in summer, shorter in winter). That’s
fine for agrarian life, less fine for coordinating cities, shipping, and later, industry.
Mechanical clocks helped standardize hours into equal units, making time less seasonal and more uniform. The ``hour'' became a stable slice rather than a
stretchy one. (It’s hard to run a railroad timetable on stretchy hours; ask any stressed-out logistics person.)
This is the philosophical shift: time stops being primarily celestial and becomes increasingly mechanical. It’s no longer just ``the sky says so,'' but
``the machine says so.''

\section*{The Longitude Problem: When Time Became Navigation’s Secret Weapon}
\textbf{I}f you want a genuinely cinematic reason to care about accurate clocks, it’s navigation.
At sea, knowing your latitude is comparatively manageable with celestial observation, but longitude historically required accurate timekeeping—because Earth’s
rotation turns time differences into position differences.
So the clock becomes a survival tool. Timekeeping stops being academic and becomes existential: get it wrong and you don’t just miss dinner—you miss land.

\subsection*{Mean Solar Time: Making the Sun Behave (on Paper)}
\textbf{F}or a long time, the fundamental unit of time was tied to Earth’s rotation: the solar day, the interval between successive returns of the Sun to a local
meridian. \cite{ref1}
But the apparent solar day varies by tens of seconds over a year due to Earth’s elliptical orbit and axial tilt, which makes ``the Sun’s time'' annoyingly
inconsistent. \cite{ref1}

\textbf{S}o we invented mean solar time—a smoothed, averaged concept that treats the day as having a constant length. \cite{ref1}

\section*{This is timekeeping’s recurring theme: reality is lumpy, so humans invent a cleaner reference frame.}
\subsection*{Greenwich and the Big Global Agreement (1884)}
\textbf{O}nce the world got more interconnected, we needed a shared reference. NASA’s historical overview notes that Greenwich Mean Time (GMT) became the reference
time standard adopted at the International Meridian Conference of 1884. \cite{ref1}

That move also standardized longitude measurement relative to the Greenwich meridian. \cite{ref1}

\textbf{E}ven if you’ve never cared about Greenwich in your life, this is a huge moment: time becomes a global coordination problem, not merely a local observation.

\subsection*{The Awkward Truth: Earth’s Rotation Is Not a Perfect Clock}
\textbf{H}ere’s the plot twist that makes metrologists sigh into their tea: Earth does not rotate with perfectly uniform speed.
NASA’s discussion explains that during the 20th century it became clear Earth’s rotational period (length of day) was gradually slowing, largely due to tidal
friction between oceans and Earth’s mantle via the Moon’s gravitational influence. \cite{ref1}
There are also shorter-term fluctuations—geophysical, climatological, core–mantle interactions—that can create significant accumulated clock error over decades. \cite{ref1}

\section*{In other words: the planet itself is a slightly unreliable drummer.}
\subsection*{Ephemeris Time: When We Fired Earth as the Boss of Time}
\textbf{I}f Earth’s spin is jittery, what do we use instead? One answer, historically, was: the orbit.

NASA’s summary notes that the International Astronomical Union introduced Ephemeris Time (ET) in 1952 as a more uniform scale for orbital calculations,
explicitly addressing the irregularities of Earth rotation-based time. \cite{ref1}
The key move was defining the second not from a day (Earth rotation) but from a year (Earth orbit)—a bigger cycle, averaged and modeled to be more stable for
astronomical work. \cite{ref1}

\section*{And this is where your question lands: the deliciously specific ``1900'' definition.}
\subsection*{The ``1900'' Second: A Year, Sliced Into a Second}
\textbf{T}he ephemeris second was defined as a fraction of the tropical year for a specific epoch around 1900, based on Newcomb’s solar tables (1895), as NASA’s
overview notes. \cite{ref1}
The canonical wording that shows up in metrology references is that the second equals the fraction \(1/31{,}556{,}925.9747\) of the tropical year for 1900 January 0 at
12 hours ephemeris time. \cite{ref1}

\textbf{I}f you’ve never met ``January 0'' before: yes, it’s a thing in certain astronomical conventions—basically the day immediately preceding January 1 (a kind of
calendrical ``zeroth day''). \cite{ref1}
The reason for anchoring to 1900 wasn’t that 1900 was magically the ``true'' year; it’s that ephemerides and astronomical models needed a defined reference
epoch, and 1900 was a practical anchor in the existing astronomical tables. \cite{ref1}

\textbf{W}ikipedia’s summary notes this definition was adopted into the SI framework around 1960 after the 1956 ephemeris-second definition work, which is the
``moment'' you’re gesturing at: a second defined via a specific tropical year tied to the 1900 epoch. \cite{ref3}

\section*{Why This Matters (Even If You Don’t Own a Telescope)}
\textbf{T}his ``1900 tropical year'' second is a gorgeous transitional fossil. It’s humanity saying:

``We love the sky.''

``We tried the spinning Earth.''

``The spinning Earth is… emotionally unstable.''

``Let’s define time using a more uniform astronomical phenomenon.'' \cite{ref1}

\textbf{A}nd if you’re building anything distributed—networks, beacons, lighthouses, synchronized systems—this is the deep ancestry of your obsession: time is a
shared contract, and the contract is only as good as the reference.

\bigskip
\hrule
\bigskip

EarthDate PDF (episode notes/source) \cite{ref2}... \emph{All calculations regarding human timings are based on a life expectancy of 80 years old (the average life expectancy in the US as of 2025 is 79.6 years...} \cite{ref2}

\textbf{H}ow do we measure and define time? how do we ensure we are consistent, how do we know how long a second is?

\subsection*{In a lifetime a person does approx. 420 million blinks — The average person blinks around 15 times a minute. Assuming the person gets 8 hours of sleep, the number of blinks in a day is 14,400}
while the number of blinks for a year is 5.3 million; 15/minute = 1 every 4 seconds. Hmmph... not much good for measuring secinds then.

\subsection*{In a lifetime a person does approx. 2.92 billion heartbeats — The heart beats about 100,000 times a day and 36 million times a year.}
100,000/day \(\approx\) 0.864 seconds between events.
not as bad... but we obviously have better solutions developed in more recent times than even heartbeats. So, what do we use?

\textbf{U}p until recently scientists still used a specific year; 1900, divided down, into 365.25 days, the days into 24 hours, then hours and minutes into 60.
That made the second \(1/86{,}400\)th of a day. \cite{ref1}

\textbf{A} second then, was measured as \(1/31{,}556{,}925.9747\)th of the year 1900. Not a very elegant solution. \cite{ref1}
So scientists looked for a new measurement. They found an isotope that changes magnetic orientation when, and only when, it’s hit with microwave energy
of a very specific frequency. This relationship was so stable, and the procedure so repeatable, that they used it to set the new standard of time: \cite{ref14}
1 second equals 9,192,631,770 cycles of the microwave energy that switches the atomic state of cesium-133. \cite{ref14}
That may not be any more elegant. But it is very precise — down to the nanosecond. So:

Atomic clock overview (background) \cite{ref3}

\section*{Going Atomic:}

\begin{verbatim}
# -------------------------------------------------------
#
# The second, symbol s, is the SI unit of time,
# defined by taking the fixed numerical value of the caesium resonant frequency,
# ΔνCs,
# the unperturbed ground-state hyperfine transition frequency of the caesium-133 atom,
# to be 9192631770 when expressed in the unit Hz, which is equal to s^−1.
#
# -------------------------------------------------------
\end{verbatim}

\section*{So, wtf does that mean for us?}
\textbf{A}n atomic clock is a clock that measures time by monitoring the resonant frequency of atoms such as caesium or rubidium. \cite{ref3}
It is based on the fact that atoms have quantised energy levels, and transitions between such levels are driven by very specific frequencies of
electromagnetic radiation. This phenomenon, since it is so accurate as a means of defining time  ot only serves to be basis for the SI definition
of the second, but also provides a very reliable, repeatable and stable source of constant clock for the purposes of measuring/displaying/announcing/dictating/
and unifying time. \cite{ref14}

\section*{The origins fo the concept:}
\textbf{P}hysicist James Clerk Maxwell proposed measuring time with the vibrations of light waves as far back as his 1873 Treatise on Electricity and Magnetism:
\emph{`A more universal unit of time might be found by taking the periodic time of vibration of the particular kind of light whose wave length is the
unit of length.'} \cite{ref4}
Maxwell argued this would be more accurate than the Earth's rotation, which defines the mean solar second for timekeeping, \cite{ref1}
and even back during the 1930s, the American physicist Isidor Isaac Rabi built equipment for atomic beam magnetic resonance frequency clocks. \cite{ref5,ref6}

\section*{The first actual build:}
\textbf{S}ince the accuracy of mechanical, electromechanical and quartz clocks is reduced by temperature fluctuations, the idea of measuring the frequency
of an atom's vibrations to keep time more accurately, as proposed by Maxwell and others years prior, the first practical atomic clock using caesium atoms
was built at the National Physical Laboratory in the United Kingdom in 1955 by Louis Essen in collaboration with Jack Parry. \cite{ref7}
The first atomic clock, the National Radio Company's 'Atomichron' sold more than 50 units during the 1950s. \cite{ref8}

\section*{OK, and?}
\begin{itemize}
\item \textbf{T}his definition underpins the system of TAI*, which is maintained by an ensemble of atomic clocks around the world. \cite{ref1}
\item The system of UTC — the basis of civil time — implements leap seconds to allow clock time to stay within one second of Earth's rotation. \cite{ref1}
\item The accurate time-keeping capabilities of atomic clocks are also used for navigation by satellite networks such as the EU’s Galileo Programme and
the United States’ GPS.
\item The timing accuracy of the atomic clocks matters because even a timing error of 1 nanosecond (\(10^{-9}\,\mathrm{s}\)) corresponds to a positional error of roughly
30 cm when multiplied by the speed of light.
\item The main variety of atomic clock in use today employs caesium atoms (or ions) cooled to near absolute zero.
For example, the United States’ primary standard, the NIST caesium fountain clock named NIST-F2, operates with a relative uncertainty around \(10^{-16}\). \cite{ref10}
\end{itemize}

\section*{Current statre of the art, optical atomic clocks.}
\textbf{T}he rapid improvement in optical atomic clock performance has prompted the global time-and-frequency community to prepare for a possible redefinition of
the SI second. \cite{ref11}
In June 2025, a coordinated international comparison of optical clocks across six countries was reported
— marking a major step towards establishing a global optical-time standard. \cite{ref11}
Optical atomic clocks are enabling new applications: ultra-precise time- and-frequency dissemination, improved global navigation satellite systems,
relativistic geodesy (measuring differences in gravitational potential via clock rates), and tests of fundamental constants and general relativity. \cite{ref12}

\section*{The definition of the second:}
\textbf{P}reviously defined as there being 31556925.9747 seconds in the tropical year 1900. \cite{ref1}, in 1968, the SI redefined the duration of the second to be 9192631770
vibrations of the unperturbed ground-state hyperfine transition frequency of the caesium-133 atom. \cite{ref16}
Further, In 1997, the International Committee for Weights
and Measures (CIPM) added that the preceding definition refers to a caesium atom at rest at a temperature of absolute zero. \cite{ref17}
Not only does ever more accurate timekeeping become necessary for the actual measurement of time itself, but also, following the 2019 revision of the SI, the
definition of every base unit except the mole and hence almost every derived unit relies on the definition of the second! \cite{ref15}
As a result, timekeeping researchers
seek an even more stable atomic reference for the second, with a plan to find a more precise definition of the second as atomic clocks improve based on
optical clocks or the Rydberg constant around 2030. \cite{ref11,ref14}

\bigskip
\hrule
\bigskip

\section*{Clock mechanism}
\textbf{A}n atomic clock is based on a system of atoms which may be in one of two possible energy states.
A group of atoms in one state is prepared, then subjected to microwave radiation.
If the radiation is of the correct frequency, a number of atoms will transition to the other energy state.
The closer the frequency is to the inherent oscillation frequency of the atoms, the more atoms will switch states.
Such correlation allows very accurate tuning of the frequency of the microwave radiation.
Once the microwave radiation is adjusted to a known frequency where the maximum number of atoms switch states, the atom and thus, its associated transition
frequency, can be used as a timekeeping oscillator to measure elapsed time. \cite{ref18}

\textbf{A}ll timekeeping devices use oscillatory phenomena to accurately measure time, whether it is the rotation of the Earth for a sundial, the swinging of
a pendulum in a grandfather clock, the vibrations of springs and gears in a watch, or voltage changes in a quartz crystal watch.
However all of these are easily affected by temperature changes and are not very accurate. The most accurate clocks use atomic vibrations to keep track of time.
Clock transition states in atoms are insensitive to temperature and other environmental factors and the oscillation frequency is much higher than any of the
other clocks (in microwave frequency regime and higher). \cite{ref3}

\section*{Error}
\textbf{O}ne of the most important factors in a clock's performance is the atomic line quality factor, Q, which is defined as the ratio of the absolute frequency
\(\nu_0\) of the resonance to the linewidth of the resonance itself \(\Delta \nu\). Atomic resonance has a much higher Q than
mechanical devices. \cite{ref3}

Atomic clocks can also be isolated from environmental effects to a much higher degree. Atomic clocks have the benefit that atoms are universal, which means
that the oscillation frequency is also universal. This is different from quartz and mechanical time measurement devices that do not have a universal frequency.

\begin{itemize}
\item A clock's quality can be specified by two parameters: \textbf{accuracy} and \textbf{stability}. \cite{ref19}
\item Accuracy is a measurement of the degree to which the clock's \emph{ticking rate} can be counted on to \emph{match some absolute standard} such as the inherent
hyperfine frequency of an isolated atom or ion. \cite{ref19}
\item Stability describes how the clock \emph{performs} when \emph{averaged over time to reduce the impact of noise} and other short-term fluctuations (see precision). \cite{ref19}
\end{itemize}

The \textbf{instability} of an atomic clock is specified by its \textbf{Allan deviation} \(\sigma_y(\tau)\). \cite{ref19}
The limiting instability due to atom or ion counting statistics is given by...
\[
\sigma_{y,\mathrm{atoms}}(\tau)\approx
\frac{\Delta \nu}{\nu_0 \sqrt{N}}
\sqrt{\frac{T_c}{\tau}}.
\]

where
\(\Delta \nu\) is the spectroscopic linewidth of the clock system,
\(N\) is the number of atoms or ions used in a single measurement,
\(T_c\) is the time required for one cycle, and
\(\tau\) is the averaging period.

This means instability is smaller when the linewidth
\(\Delta \nu\) is smaller and when
\(\sqrt{N}\) (the signal to noise ratio) is larger. The stability improves as the time \(\tau\) over which the measurements are
averaged increases from seconds to hours to days. \cite{ref19}

\subsection*{The stability is most heavily affected by the oscillator frequency \(\nu_0\).}
This is why optical clocks such as strontium clocks (429 terahertz) are much more stable than caesium clocks (9.19 GHz). \cite{ref22}

\textbf{M}odern clocks such as atomic fountains or optical lattices that use sequential interrogation are found to generate type of noise that mimics and adds to the
instability inherent in atom or ion counting.
This effect is called the Dick effect and is typically the primary stability limitation for the newer atomic clocks. \cite{ref20}
It is an aliasing effect; high frequency noise components in the local oscillator (``LO'') are heterodyned to near zero frequency by harmonics of the repeating
variation in feedback sensitivity to the LO frequency. \cite{ref20}
The effect places new and stringent requirements on the LO, which must now have low phase noise in addition to high stability, thereby increasing the cost and
complexity of the system. \cite{ref22}

\subsection*{For the case of an LO with Flicker frequency noise where \(\sigma_y^{\mathrm{LO}}(\tau)\) is independent of \(\tau\), the interrogation time is \(T_i\), and where the duty factor \(d=T_i/T_c\) has typical values \(0.4<d<0.7\), the Allan deviation can be approximated as}
\[
\sigma_{y,\mathrm{Dick}}(\tau)\approx
\frac{\sigma_y^{\mathrm{LO}}}{\sqrt{2\ln(2)}}
\left|\frac{\sin(\pi d)}{\pi d}\right|
\sqrt{\frac{T_c}{\tau}}.
\] \cite{ref21,ref23}

This expression shows the same dependence on
\(T_c/\tau\) as does
\(\sigma_{y,\mathrm{atoms}}(\tau)\), and, for many of the newer clocks, is significantly larger. \cite{ref22}

\textbf{A}nalysis of the effect and its consequence as applied to optical standards has been treated in a major review (Ludlow, et al., 2015) that lamented on
``the pernicious influence of the Dick effect'', and in several other papers. \cite{ref22,ref24,ref25}

\textbf{T}he accuracy of atomic clocks has improved continuously since the first prototype in the 1950s. The first generation of atomic clocks were based on measuring caesium, rubidium,
and hydrogen atoms. The International Bureau of Weights and Measures (BIPM) provides a list of frequencies that serve as secondary representations of the
second. This list contains the frequency values and respective standard uncertainties for the rubidium microwave transition and other optical transitions,
including neutral atoms and single trapped ions. These secondary frequency standards can be as accurate as one part in \(10^{18}\); however, the uncertainties in the
list are one part in \(10^{14}\)–\(10^{16}\). This is because the uncertainty in the central caesium standard against which the secondary standards are calibrated is one
part in \(10^{14}\)–\(10^{16}\). \cite{ref28,ref29}

\bigskip
\hrule
\bigskip

\section*{Rubidium standard}
Rubidium standard overview \cite{ref30}

\textbf{A} rubidium standard or rubidium atomic clock is a frequency standard in which a specified hyperfine transition of electrons in rubidium-87 atoms is used to
control the output frequency. \cite{ref30}

The Rb standard is the most inexpensive, compact, and widely produced atomic clock, used to control the frequency of television stations, cell phone base
stations, in test equipment, and global navigation satellite systems like GPS. Commercial rubidium clocks are less accurate than caesium atomic clocks,
which serve as primary frequency standards, so a rubidium clock is usually used as a secondary frequency standard. \cite{ref30}

\emph{(fig.x. fig-x-Rubidium-oscillator-wikicommons.jpg)}

\textbf{C}ommercial rubidium frequency standards operate by disciplining a crystal oscillator to the rubidium hyperfine transition of 6.8 GHz (6834682610.904 Hz).
The intensity of light from a rubidium discharge lamp that reaches a photodetector through a resonance cell will drop by about 0.1\% when the rubidium vapor in
the resonance cell is exposed to microwave power near the transition frequency. The crystal oscillator is stabilized to the rubidium transition by detecting
the light dip while sweeping an RF synthesizer (referenced to the crystal) through the transition frequency. \cite{ref30}

\subsection*{About Rubidium standards:}
BIPM recommended secondary representations / standard frequencies \cite{ref29}

The radiations listed below may be used recommended values of standard frequencies as secondary representations of the second:
Standard Frequency [SRS]
6.835 GHz  –  87Rb
Update: 2021 \cite{ref29}

Pavillon de Breteuil F-92312 Sèvres Cedex FRANCE Copyright @ BIPM tous droits réservés

\bigskip
\hrule
\bigskip

\subsection*{The BIPM defines the unperturbed ground-state hyperfine transition frequency of the rubidium-87 atom, 6 834 682 610.904 312 6 Hz, in terms of the caesium standard frequency.}
\textbf{A}tomic clocks based on rubidium standards are therefore regarded as secondary representations of the second. \cite{ref29}

\textbf{T}he advantages of rubidium atomic clocks are their low cost, small size (commercial standards are as small as \(1.7\times 10^5\,\mathrm{mm}^3\)) and short-term stability.
They are used in many commercial, portable and aerospace applications. Modern rubidium standard tubes last more than ten years, and can cost as little as US\$50.
Some commercial applications use a rubidium standard periodically corrected by a global positioning system receiver (see GPS disciplined oscillator).
This achieves excellent short-term accuracy, with long-term accuracy equal to (and traceable to) the US national time standards. \cite{ref30,ref34}

\textbf{A} list of frequencies recommended for secondary representations of the second is maintained by the International Bureau of Weights and Measures (BIPM) since
2006 and is available online. The list contains the frequency values and the respective standard uncertainties for the rubidium microwave transition and for
several optical transitions. These secondary frequency standards are accurate at the level of \(10^{-18}\); however, the uncertainties provided in the list are in
the range \(10^{-14}\) – \(10^{-15}\) since they are limited by the linking to the caesium primary standard that currently (2018) defines the second. \cite{ref28,ref29}

\begin{verbatim}
# -----------------------------------------------------
# Type Working frequency (Hz) Relative Allan deviation
# (typical clocks) Reference
# 133Cs 9.192631770×10^9 by definition 10−13
#
# 87Rb 6.8346826109043126×10^9 10−12
#
# ----------------------------------------------------
\end{verbatim}

\bigskip
\hrule
\bigskip

GPS disciplined oscillator overview \cite{ref33}

\section*{GPS disciplined oscillator}
\textbf{A} GPS clock, or GPS disciplined oscillator (GPSDO), is a combination of a GPS receiver and a high-quality, stable oscillator such as a quartz or rubidium
oscillator whose output is controlled to agree with the signals broadcast by GPS or other GNSS satellites. \cite{ref33}
GPSDOs work well as a source of timing because the satellite time signals must be accurate in order to provide positional accuracy for GPS in navigation.
These signals are accurate to nanoseconds and provide a good reference for timing applications. \cite{ref33}

\emph{(fig.y. GPDSO)}

\subsection*{Applications}
\textbf{G}PSDOs serve as an indispensable source of timing in a range of applications, and some technology applications would not be practical without them. \cite{ref33}
GPSDOs are used as the basis for Coordinated Universal Time (UTC) around the world. UTC is the official accepted standard for time and frequency.
UTC is controlled by the International Bureau of Weights and Measures (BIPM). Timing centers around the world use GPS to align their own time scales to UTC. \cite{ref1,ref33}
GPS based standards are used to provide synchronization to wireless base stations and serve well in standards laboratories as an alternative to cesium-based
references. \cite{ref34}
GPSDOs can be used to provide synchronization of multiple RF receivers, allowing for RF phase coherent operation among the receivers and
applications, such as passive radar and ionosondes. \cite{ref33}

\section*{So, the litehaus uses a GPSDO - why?}
\subsection*{GPSDO Operation}
\textbf{A} GPSDO works by disciplining, or steering a high-quality quartz or rubidium oscillator by locking the output to a GPS signal via a tracking loop.
The disciplining mechanism works in a similar way to a phase-locked loop (PLL), but in most GPSDOs the loop filter is replaced with a microcontroller that
uses software to compensate for not only the phase and frequency changes of the local oscillator, but also for the ``learned'' effects of aging, temperature, and
other environmental parameters. \cite{ref33}

\textbf{O}ne of the keys to the usefulness of a GPSDO as a timing reference is the way it is able to combine the stability characteristics of the GPS signal and the
oscillator controlled by the tracking loop. GPS receivers have excellent long-term stability (as characterized by their Allan deviation) at averaging times
greater than several hours. However, their short-term stability is degraded by limitations of the internal resolution of the one pulse per second (1PPS)
reference timing circuits, signal propagation effects such as multipath interference, atmospheric conditions, and other impairments. On the other hand,
a quality oven-controlled oscillator has better short-term stability but is susceptible to thermal, aging, and other long-term effects.
A GPSDO aims to utilize the best of both sources, combining the short-term stability performance of the oscillator with the long-term stability of the GPS
signals to give a reference source with excellent overall stability characteristics. \cite{ref34}

\textbf{G}PSDOs typically phase-align the internal flywheel oscillator to the GPS signal by using dividers to generate a 1PPS signal from the reference oscillator,
then phase comparing this 1PPS signal to the GPS-generated 1PPS signal and using the phase differences to control the local oscillator frequency in small
adjustments via the tracking loop. \cite{ref33}
This differentiates GPSDOs from their cousins NCOs (numerically controlled oscillator). Rather than disciplining an
oscillator via frequency adjustments, NCOs typically use a free-running, low-cost crystal oscillator and adjust the output phase by digitally lengthening or
shortening the output phase many times per second in large phase steps, assuring that on average the number of phase transitions per second is aligned to
the GPS receiver reference source. This guarantees frequency accuracy at the expense of high phase noise and jitter, a degradation that true GPSDOs do not suffer. \cite{ref33}

\begin{itemize}
\item When the GPS signal becomes unavailable, the GPSDO goes into a state of holdover, where it tries to maintain accurate timing using only the internal oscillator. \cite{ref33}
\item Sophisticated algorithms are used to compensate for the aging and temperature stability of the oscillator while the GPSDO is in holdover. \cite{ref35}
\end{itemize}

\textbf{T}he use of Selective Availability (SA) prior to May 2000 restricted the accuracy of GPS signals available for civilian use and in turn presented challenges to
the accuracy of GPSDO-derived timing. The turning off of SA resulted in a significant increase in the accuracy that GPSDOs can offer. \cite{ref33}
GPSDOs are capable
of generating frequency accuracies and stabilities on the order of parts per billion for even entry-level, low-cost units, to parts per trillion for more
advanced units within minutes after power-on, and are thus one of the highest-accuracy physically-derived reference standards available. \cite{ref34}

\bigskip
\hrule
\bigskip

\emph{*TAI is not distributed in everyday timekeeping.\\
Instead, an integer number of leap seconds are added or subtracted to correct for the Earth's rotation, producing UTC.\\
The number of leap seconds is changed so that mean solar noon at the prime meridian (Greenwich) does not deviate from UTC noon by more than 0.9 seconds.*} \cite{ref1}

\bigskip
\hrule
\bigskip

\section*{Notes / claims}
\begin{enumerate}[label=\alph*.]
\item Researchers at the University of Wisconsin-Madison have demonstrated a clock that will not lose a second in 300 billion years. \cite{ref35}
\item One second in 13.8 billion years, the age of the universe, is an accuracy of \(2.3\times 10^{-18}\).
\end{enumerate}

\bigskip
\hrule
\bigskip

\section*{References}
\begin{thebibliography}{99}

\bibitem{ref1}
F. Espenak, ``Terrestrial Dynamical Time,'' \emph{NASA Eclipse Web Site}. [Online]. Available: \url{https://eclipse.gsfc.nasa.gov/LEcat5/time.html}. Accessed: 2025-12-25.

\bibitem{ref2}
\emph{EarthDate Episode 192: ``C''} (PDF), EarthDate. [Online]. Available: \url{https://www.earthdate.org/files/000/002/254/EarthDate_192_C.pdf}. Accessed: 2025-12-25.

\bibitem{ref3}
``Atomic clock,'' \emph{Wikipedia}. [Online]. Available: \url{https://en.wikipedia.org/wiki/Atomic_clock}. Accessed: 2025-12-25.

\bibitem{ref4}
F. Achard, ``James Clerk Maxwell, \emph{A treatise on electricity and magnetism}, first edition (1873),'' in \emph{Landmark Writings in Western Mathematics 1640–1940}. Elsevier, 2005, pp. 564–587, doi: 10.1016/B978-044450871-3/50125-X.

\bibitem{ref5}
I. I. Rabi, ``Space quantization in a gyrating magnetic field,'' \emph{Phys. Rev.}, vol. 51, no. 8, pp. 652–654, 1937, doi: 10.1103/PhysRev.51.652.

\bibitem{ref6}
I. I. Rabi, J. R. Zacharias, S. Millman, and P. Kusch, ``A new method of measuring nuclear magnetic moment,'' \emph{Phys. Rev.}, vol. 53, no. 4, p. 318, 1938, doi: 10.1103/PhysRev.53.318.

\bibitem{ref7}
L. Essen and J. V. L. Parry, ``An atomic standard of frequency and time interval: A cæsium resonator,'' \emph{Nature}, vol. 176, no. 4476, p. 280, 1955, doi: 10.1038/176280a0.

\bibitem{ref8}
``Atomichron,'' \emph{Wikipedia}. [Online]. Available: \url{https://en.wikipedia.org/wiki/Atomichron}. Accessed: 2025-12-25.

\bibitem{ref9}
National Institute of Standards and Technology (NIST), ``NIST launches a new U.S. time standard: NIST-F2 atomic clock,'' NIST, 2014-04-03. [Online]. Available: \url{https://www.nist.gov/news-events/news/2014/04/nist-launches-new-us-time-standard-nist-f2-atomic-clock}. Accessed: 2025-12-25.

\bibitem{ref10}
T. P. Heavner \emph{et al.}, ``First accuracy evaluation of NIST-F2,'' \emph{Metrologia}, vol. 51, no. 3, pp. 174–182, 2014, doi: 10.1088/0026-1394/51/3/174.

\bibitem{ref11}
N. Dimarcq \emph{et al.}, ``Roadmap towards the redefinition of the second,'' \emph{Metrologia}, 2024. [Online]. Available: \url{https://eprintspublications.npl.co.uk/9977/1/eid9977.pdf}. Accessed: 2025-12-25.

\bibitem{ref12}
National Institute of Standards and Technology (NIST), ``Optical Clocks: The Future of Time,'' NIST, 2025-07-14. [Online]. Available: \url{https://www.nist.gov/atomic-clocks/how-atomic-clocks-work/optical-clocks-future-time}. Accessed: 2025-12-25.

\bibitem{ref13}
D. D. McCarthy and P. K. Seidelmann, \emph{TIME—From Earth Rotation to Atomic Physics}. Weinheim, Germany: Wiley‑VCH, 2009.

\bibitem{ref14}
Bureau International des Poids et Mesures (BIPM), ``SI base unit: second (s),'' BIPM. [Online]. Available: \url{https://www.bipm.org/en/si-base-units/second}. Accessed: 2025-12-25.

\bibitem{ref15}
Bureau International des Poids et Mesures (BIPM), \emph{The International System of Units (SI Brochure)}, 9th ed., PDF. [Online]. Available: \url{https://www.bipm.org/documents/20126/41483022/SI-Brochure-9-EN.pdf}. Accessed: 2025-12-25.

\bibitem{ref16}
Bureau International des Poids et Mesures (BIPM), ``Resolution 1 of the 13th CGPM (1967),'' BIPM. [Online]. Available: \url{https://www.bipm.org/en/committees/cg/cgpm/13-1967/resolution-1}. Accessed: 2025-12-25.

\bibitem{ref17}
Innovation and Technology Commission (ITC), ``Time (the second, s),'' Government of the Hong Kong SAR, 2018-12-31. [Online]. Available: \url{https://www.itc.gov.hk/en/quality/scl/teachers_students/si/second.html}. Accessed: 2025-12-25.

\bibitem{ref18}
Timeanddate.com, ``How do atomic clocks work?'' [Online]. Available: \url{https://www.timeanddate.com/time/how-do-atomic-clocks-work.html}. Accessed: 2025-12-25.

\bibitem{ref19}
D. W. Allan, ``Statistics of atomic frequency standards,'' \emph{Proc. IEEE}, vol. 54, no. 2, pp. 221–230, Feb. 1966.

\bibitem{ref20}
G. J. Dick, ``Local oscillator induced instabilities in trapped ion frequency standards,'' in \emph{Proc. Precise Time and Time Interval (PTTI) Meeting}, 1987.

\bibitem{ref21}
J. A. Barnes \emph{et al.}, \emph{Characterization of Frequency Stability}, NBS Technical Note 394, 1970.

\bibitem{ref22}
A. D. Ludlow, M. M. Boyd, J. Ye, E. Peik, and P. O. Schmidt, ``Optical atomic clocks,'' \emph{Rev. Mod. Phys.}, vol. 87, no. 2, pp. 637–701, 2015, doi: 10.1103/RevModPhys.87.637.

\bibitem{ref23}
G. Santarelli \emph{et al.}, ``Frequency stability degradation of an oscillator slaved to a periodically interrogated atomic resonator,'' \emph{IEEE Trans. Ultrason. Ferroelectr. Freq. Control}, vol. 45, no. 4, pp. 887–894, 1998, doi: 10.1109/58.710548.

\bibitem{ref24}
A. Quessada \emph{et al.}, ``The Dick effect for an optical frequency standard,'' \emph{J. Opt. B: Quantum Semiclass. Opt.}, vol. 5, no. 2, pp. S150–S154, 2003, doi: 10.1088/1464-4266/5/2/373.

\bibitem{ref25}
P. G. Westergaard, J. Lodewyck, and P. Lemonde, ``Minimizing the Dick effect in an optical lattice clock,'' \emph{IEEE Trans. Ultrason. Ferroelectr. Freq. Control}, vol. 57, no. 3, pp. 623–628, 2010, doi: 10.1109/TUFFC.2010.1457.

\bibitem{ref26}
N. Poli, ``Optical atomic clocks,'' \emph{La Rivista del Nuovo Cimento}, vol. 36, no. 12, p. 555, 2014, doi: 10.1393/ncr/i2013-10095-x.

\bibitem{ref27}
Bureau International des Poids et Mesures (BIPM), ``Mise en pratique for the definition of the second in the SI,'' 2019-05-20. [Online]. Available: \url{https://www.bipm.org/en/publications/mises-en-pratique}. Accessed: 2025-12-25.

\bibitem{ref28}
Bureau International des Poids et Mesures (BIPM), ``Standard frequencies for applications including the recommended secondary representations of the second,'' BIPM. [Online]. Available: \url{https://www.bipm.org/en/publications/mises-en-pratique/standard-frequencies-second}. Accessed: 2025-12-25.

\bibitem{ref29}
``Rubidium standard,'' \emph{Wikipedia}. [Online]. Available: \url{https://en.wikipedia.org/wiki/Rubidium_standard}. Accessed: 2025-12-25.

\bibitem{ref30}
W. J. Riley, ``A history of the rubidium frequency standard,'' Dec. 2019. (Use your local PDF / archived source.)

\bibitem{ref31}
National Physical Laboratory (NPL), ``OC18,'' 2019. (Use your local source / NPL page if you have it.)

\bibitem{ref33}
``GPS disciplined oscillator,'' \emph{Wikipedia}. [Online]. Available: \url{https://en.wikipedia.org/wiki/GPS_disciplined_oscillator}. Accessed: 2025-12-25.

\bibitem{ref34}
M. A. Lombardi, ``The use of GPS disciplined oscillators as primary frequency standards for calibration and metrology laboratories,'' \emph{NCSLI Measure}, vol. 3, no. 3, pp. 56–65, Sep. 2008, doi: 10.1080/19315775.2008.11721437.

\bibitem{ref35}
University of Wisconsin–Madison, Department of Physics, ``Ultraprecise atomic clock poised for new physics discoveries,'' 2022-02-16. [Online]. Available: \url{https://www.physics.wisc.edu/2022/02/16/ultraprecise-atomic-clock-poised-for-new-physics-discoveries/}. Accessed: 2025-12-25.

\end{thebibliography}

\end{document}
